\documentclass[12pt]{article}

\usepackage[margin=1in]{geometry}
\usepackage{amsmath,amsthm,amssymb}
\usepackage{fancyhdr}
\usepackage[small,compact]{titlesec}
\usepackage{float}

\lhead{Erich Menge}
\chead{\classnameandsection}
\rhead{\homeworktitle}

\pagestyle{fancy}

\newcommand{\sethomeworknumber}[1]{
  \newcommand{\homeworktitle}{Homework #1}
}

\newcommand{\N}{\mathbb{N}}
\newcommand{\Z}{\mathbb{Z}}
\newcommand{\homeworkheader}[1]{
  \title{\vspace{2in}\homeworktitle}
  \author{Erich Menge (X.500: menge053, Student ID: 4624713) \\
  #1}
  \maketitle
  \newpage
}

\newenvironment{problem}[1]{
  \ignorespaces
  \section*{Problem #1}
}{
  \ignorespacesafterend
}

\newenvironment{solution}{
  \ignorespaces
  \subsection*{Solution}
}{
  \ignorespacesafterend
}

\newcommand{\classnameandsection}{CSCI 4011 Formal Languages And Automata Theory Section 3}

\usepackage{listings}

\sethomeworknumber{4}

\begin{document}

\homeworkheader{\classnameandsection}

\begin{problem}{9.2-3}
  Write an iterative version of RANDOMIZED-SELECT.
  \begin{solution}
    I'll implement this in C because I don't use C enough and could use the practice.
    \begin{figure}[H]
      \centering
      \caption{C iterative randomized select}
      \lstinputlisting{problem_1.c}
    \end{figure}
  \end{solution}
\end{problem}

\begin{problem}{9.3-3}
  Show how quicksort can be made to run in $O(n \lg n)$ time in the worst case, assuming that all elements are
  distinct.
  \begin{solution}
    From chapter 7.2 we know that the best-case of quick-sort is $\Theta(n \lg n)$ when partitioned evenly from the
    recurrence $T(n) = 2T(n/2) + \Theta(n)$. So in order to build a quick-sort that runs in $O(n \lg n)$ time we need to
    ensure the partitions are split evenly.

    Since the select algorithm runs in worst case linear time $O(n)$, using that to find the median for use in
    partitionioning instead of the method in chapter 7 will replace $\Theta(n)$ with $O(n)$ giving the recurrence $T(n)
    = 2T(n/2) + O(n) = O(n \lg n)$.
    \begin{figure}[H]
      \centering
      \caption{C implementation of quicksort with select}
      \lstinputlisting{problem_2.c}
    \end{figure}
  \end{solution}
\end{problem}

\begin{problem}{9.3-5}
  Suppose that you have a ``black-box'' worst-case linear-time median subroutine. Give a simple, linear-time algorithm
  that solves the selection problem for an arbitrary order statistic.
  \begin{solution}
    \begin{figure}[H]
      \centering
      \caption{C++ Implementation of linear-time select}
      \lstinputlisting{problem_9_3_5.cpp}
    \end{figure}
    Since the partition algorithm from prior chapters is $O(n)$ and the ``black-box'' median operation is $O(n)$, and
    the select algorithm runs in linear time the total runtime is $O(n)$.
  \end{solution}
\end{problem}

\begin{problem}{9.3-8}
  Let $X[1\ldots n]$ and $Y[1\ldots n]$ be two arrays, each containing $n$ numbers already in sorted order. Give an
  $O(\lg n)$-time algorithm to find the median of all $2n$ elements in arrays $X$ and $Y$.
  \begin{solution}
    To get $O(\lg n)$ we need to cut the data set in (approximately) half each time through the code. Below is a Ruby
    implementation. The median of each list compared. Since the lists are in order, this is simple arithmetic and takes
    constant time. Then we drop half of each list, the small values from one, the large values from the other. These
    operations are all done in constant time. Finally we recurse passing our new halved lists until we are left with a
    final two to compare. One in each list. Since the size of the data set is halved each time, it is $O(\lg n)$.
    \begin{figure}[H]
      \centering
      \caption{A Ruby Implementation}
      \lstinputlisting{problem_9_3_8.rb}
    \end{figure}
  \end{solution}
\end{problem}

\begin{problem}{21.3-2}
  Write a nonrecursive version of FIND-SET with path compression.
  \begin{solution}
    \begin{figure}[H]
      \centering
      \caption{C++ iterative find-set}
      \lstinputlisting{problem_21_3_2.cpp}
    \end{figure}
  \end{solution}
\end{problem}

\begin{problem}{21.3-3}
  Give a sequence of $m$ MAKE-SET, UNION, and FIND-SET operations, $n$ of which are MAKE-SET operations, that takes
  $\Omega(m \lg n)$ time when we use union by rank only.
  \begin{solution}
    For MAKE-SET we have $n$ operations to make each single item set. We make binary trees out of the nodes using UNION
    with an $n-1$ operations. So for the remaining FIND-SET operations we need $(m - n - (n - 1)) = m -2n + 1$
    operations. Since the height of the tree is $\lfloor \lg(n) \rfloor$ we have $(m - 2n + 1)\lg(n)$. Where $m$ is
    sufficiently large $m - 2n + 1 = \Omega(m\lg(n))$ for $m \ge 3n$.
  \end{solution}
\end{problem}

\end{document}

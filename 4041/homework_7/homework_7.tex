\documentclass[12pt]{article}

\usepackage[margin=1in]{geometry}
\usepackage{amsmath,amsthm,amssymb}

\newcommand{\N}{\mathbb{N}}
\newcommand{\Z}{\mathbb{Z}}
\newcommand{\homeworkheader}[2]{
  \title{Homework #1}
  \author{Erich Menge (X.500: menge053, Student ID: 4624713) \\
  #2}
  \maketitle
}

\newenvironment{problem}[1]{
  \ignorespaces
  \section*{Problem #1}
}{
  \ignorespacesafterend
}

\newcommand{\classnameandsection}{CSCI 4707 Practice of Database Systems}


\sethomeworknumber{7}

\begin{document}

\homeworkheader{\classnameandsection}

\begin{problem}{16.2-5}
  Describe an efficient algorithm that, given a set $\{ x_1,x_2,\ldots, x_n \}$ of points on the real line, determines
  the smallest set of unit-length closed intervals that contains all of the given points. Argue that your algorithm is
  correct.
\end{problem}

\begin{problem}{16.2-7}
  Suppose you are given two sets $A$ and $B$, each containing n positive integers. You can choose to reorder each set
however you like. After reordering, let $a_i$ be the $i$th element of set $A$, and let $b$ be the $i$th element of set
B. You then receive a payoff $\displaystyle\prod\limits_{i = 1}^n a_i^{b_i}$ . Give an algorithm that will maximize your
   payoff. Prove that your algorithm maximizes the payoff, and state its running time.
\end{problem}

\begin{problem}{16.3-2}
  Prove that a binary tree that is not full cannot correspond to an optimal prefix code.
\end{problem}

\begin{problem}{16.3-3}
  What is an optimal Huffman code for the following set of frequencies, based on the first 8 Fibonacci numbers?
  \br
  a:1 b:1 c:2 d:3 e:5 f:8 g:13 h:21
  \br
  Can you generalize your answer to find the optimal code when the frequencies are the first n Fibonacci numbers?
\end{problem}


\end{document}

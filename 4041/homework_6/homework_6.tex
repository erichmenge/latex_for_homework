\documentclass[12pt]{article}

\usepackage[margin=1in]{geometry}
\usepackage{amsmath,amsthm,amssymb}

\newcommand{\N}{\mathbb{N}}
\newcommand{\Z}{\mathbb{Z}}
\newcommand{\homeworkheader}[2]{
  \title{Homework #1}
  \author{Erich Menge (X.500: menge053, Student ID: 4624713) \\
  #2}
  \maketitle
}

\newenvironment{problem}[1]{
  \ignorespaces
  \section*{Problem #1}
}{
  \ignorespacesafterend
}

\newcommand{\classnameandsection}{CSCI 4707 Practice of Database Systems}


\sethomeworknumber{6}

\begin{document}

\homeworkheader{\classnameandsection}

\begin{problem}{15.2-3}
  Use the substitution method to show that the solution to the recurrence (15.6) is $\Omega(2^n)$.
\end{problem}

\begin{problem}{15.3-6}
  Imagine that you wish to exchange one currency for another. You realize that instead of directly exchanging one
  currency for another, you might be better off making a series of trades through other currencies, winding up with the
  currency you want. Suppose that you can trade $n$ different currencies, numbered $1, 2, \ldots, n$, where you start
  with currency 1 and wish to wind up with currency $n$. You are given, for each pair of currencies $i$ and $j$ , an
  exchange rate $r_{ij}$, meaning that if you start with $d$ units of currency $i$, you can trade for $dr_{ij}$ units of
  currency $j$. A sequence of trades may entail a commission, which depends on the number of trades you make. Let $c_k$
  be the commission that you are charged when you make $k$ trades. Show that,if $c_k = 0$ for all $k = 1,2
  \dots,n$,then the problem of finding the best sequence of exchanges from currency 1 to currency $n$ exhibits optimal
  substructure. Then show that if commissions ck are arbitrary values, then the problem of finding the best sequence of
  exchanges from currency 1 to currency n does not necessarily exhibit optimal substructure.
\end{problem}

\begin{problem}{16.1-2}
  Suppose that instead of always selecting the first activity to finish, we instead select the last activity to start
  that is compatible with all previously selected activities. Describe how this approach is a greedy algorithm, and
  prove that it yields an optimal solution.
  \begin{solution}
    This approach is greedy in the same way that the first approach is greedy. This time we're starting from the end and
    working toward the beginning of the timeline. We have two cases for optimal solution. A solution that contains the
    greedy choice and one that doesn't. In the case we have a solution without the greedy choice we simply swap the last
    activity to start with one that starts later.

    This then has an optimal substructure just like the first approach. We take a list of sorted events, and find the
    greedy choice which is the last activity to start. Then the optimal structure is the solution to the subproblem with
    the greedy choice added back on to it so that there are no overlapping activities. The subproblem is solved the same
    way, so that each subproblem has a greedy choice and optimal structure. This is an inductive proof.

    This works the same was as the original problem, only going the other direction (from the end of the timeline to the
    beginning). It exhibits the same optimal substructure.
  \end{solution}
\end{problem}

\begin{problem}{16.1-3}
  Not just any greedy approach to the activity-selection problem produces a maximum-size set of mutually compatible
  activities. Give an example to show that the approach of selecting the activity of least duration from among those
  that are compatible with previously selected activities does not work. Do the same for the approaches of always
  selecting the compatible activity that overlaps the fewest other remaining activities and always selecting the
  compatible remaining activity with the earliest start time.
  \begin{solution}
    Below is a Ruby implementation that demonstrates selecting the earliest start time gets you more events than the
    other methods. The reason why the other methods fail is because they do not exhibit an optimal substructure where
    you can select a greedy choice each subproblem. The shortest durating tells you nothing about the placement of the
    activities so there is no optimal substructure. The same is true with the events with fewest overlaps. There is no
    structure that actually places the activities in some optimal way.
    \lstinputlisting[breaklines=true]{16_1_3.rb}
  \end{solution}
\end{problem}

\end{document}

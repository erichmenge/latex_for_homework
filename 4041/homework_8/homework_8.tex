\documentclass[12pt]{article}

\usepackage[margin=1in]{geometry}
\usepackage{amsmath,amsthm,amssymb}

\newcommand{\N}{\mathbb{N}}
\newcommand{\Z}{\mathbb{Z}}
\newcommand{\homeworkheader}[2]{
  \title{Homework #1}
  \author{Erich Menge (X.500: menge053, Student ID: 4624713) \\
  #2}
  \maketitle
}

\newenvironment{problem}[1]{
  \ignorespaces
  \section*{Problem #1}
}{
  \ignorespacesafterend
}

\newcommand{\classnameandsection}{CSCI 4707 Practice of Database Systems}


\sethomeworknumber{8}

\begin{document}

\homeworkheader{\classnameandsection}

\begin{problem}{16.3-6}
  Suppose we have an optimal prefix code on a set $C = \{0,1,\ldots,n-1\}$ of characters and we wish to transmit this
  code using as few bits as possible. Show how to represent any optimal prefix code on $C$ using only $2n - 1 + n \lceil
  \lg n \rceil$ bits. (Hint: Use $2n - 1$ bits to specify the structure of the tree, as discovered by a walk of the
  tree.)
\end{problem}

\begin{problem}{16.3-7}
  Generalize Huffman's algorithm to ternary codewords (i.e., codewords using the symbols 0, 1, and 2), and prove that it
  yields optimal ternary codes.
\end{problem}

\begin{problem}{16.3-8}
  Suppose that a data file contains a sequence of 8-bit characters such that all 256 characters are about equally
  common: the maximum character frequency is less than twice the minimum character frequency. Prove that Huffman coding
  in this case is no more efficient than using an ordinary 8-bit fixed-length code.
\end{problem}

\begin{problem}{23.1-1}
  Let $(u,v)$ be a minimum-weight edge in a connected graph G. Show that $(u,v)$ belongs to some minimum spanning tree
  of G.
\end{problem}

\begin{problem}{23.1-5}
  Let $e$ be a maximum-weight edge on some cycle of connected graph $G = (V,E)$. Prove that there is a minimum spanning
  tree of $G' = (V,E - \{e\})$ that is also a minimum spanning tree of G. That is, there is a minimum spanning tree of G
  that does not include $e$.
\end{problem}

\end{document}

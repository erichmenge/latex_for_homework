\documentclass[12pt]{article}

\usepackage[margin=1in]{geometry}
\usepackage{amsmath,amsthm,amssymb}
\usepackage{fancyhdr}
\usepackage[small,compact]{titlesec}
\usepackage{float}

\lhead{Erich Menge}
\chead{\classnameandsection}
\rhead{\homeworktitle}

\pagestyle{fancy}

\newcommand{\sethomeworknumber}[1]{
  \newcommand{\homeworktitle}{Homework #1}
}

\newcommand{\N}{\mathbb{N}}
\newcommand{\Z}{\mathbb{Z}}
\newcommand{\homeworkheader}[1]{
  \title{\vspace{2in}\homeworktitle}
  \author{Erich Menge (X.500: menge053, Student ID: 4624713) \\
  #1}
  \maketitle
  \newpage
}

\newenvironment{problem}[1]{
  \ignorespaces
  \section*{Problem #1}
}{
  \ignorespacesafterend
}

\newenvironment{solution}{
  \ignorespaces
  \subsection*{Solution}
}{
  \ignorespacesafterend
}

\newcommand{\classnameandsection}{CSCI 4011 Formal Languages And Automata Theory Section 3}


\sethomeworknumber{8}

\begin{document}

\homeworkheader{\classnameandsection}

\begin{problem}{16.3-6}
  Suppose we have an optimal prefix code on a set $C = \{0,1,\ldots,n-1\}$ of characters and we wish to transmit this
  code using as few bits as possible. Show how to represent any optimal prefix code on $C$ using only $2n - 1 + n \lceil
  \lg n \rceil$ bits. (Hint: Use $2n - 1$ bits to specify the structure of the tree, as discovered by a walk of the
  tree.)
\end{problem}

\begin{problem}{16.3-7}
  Generalize Huffman's algorithm to ternary codewords (i.e., codewords using the symbols 0, 1, and 2), and prove that it
  yields optimal ternary codes.
  \begin{solution}
    \begin{lstlisting}
      HUFFMAN(C)
        n = |C|
        Q = C
        for i = 1 to n - 1
          allocate a new node z
          z.left = w = Extract-Min(Q)
          z.mid = x = Extract-Min(Q)
          z.right = y = Extract-Min(Q)
          z.freq = x.freq + y.freq + z.freq
          Insert(Q, z)
        return Extract-Min(Q)
    \end{lstlisting}
    \begin{proof}
      This modified algorithim works the same way as the original, only using a ternary tree instead of a binary tree.
      We can still construct the tree in such a way that the minimum frequency characters lie at the bottom of this
      ternary tree using a modified priority queue that works on ternary trees. Then the greedy choice is the root of
      this lowest freqency sub-tree with its three child nodes removed. Which means the optimal solution is the solution
      to the remaining tree with the greedy choice added back on.  This is done all the way up the tree as before.
    \end{proof}
  \end{solution}
\end{problem}

\begin{problem}{16.3-8}
  Suppose that a data file contains a sequence of 8-bit characters such that all 256 characters are about equally
  common: the maximum character frequency is less than twice the minimum character frequency. Prove that Huffman coding
  in this case is no more efficient than using an ordinary 8-bit fixed-length code.
\end{problem}

\begin{problem}{23.1-1}
  Let $(u,v)$ be a minimum-weight edge in a connected graph G. Show that $(u,v)$ belongs to some minimum spanning tree
  of G.
  \begin{solution}
    We can assume that $(u,v)$ does not belong to some minimum spanning tree. Then starting at the root node we add an
    edge for the greedy choice $e$ where $w(e) \le w(e')$.  Since $w(e)$ is less than or equal to the old edge, if we
    cut the old edge out we have a path with less weight.  When we come to the node $u$ which is connected to $v$ we
    find that it is connected by a minimum weight edge. But $w(e) \le w(u,v)$, so we know that the greedy choice
    selection of the MST must be no heavier than the existing edge. We assume that $(u,v)$ does not belong to this
    MST.  But it has to because it is a minimum weight edge and the greedy choice can be no greater than it.  This is a
    contradiction. So $(u,v)$ must belong to this minimum spanning tree.
  \end{solution}
\end{problem}

\begin{problem}{23.1-5}
  Let $e$ be a maximum-weight edge on some cycle of connected graph $G = (V,E)$. Prove that there is a minimum spanning
  tree of $G' = (V,E - \{e\})$ that is also a minimum spanning tree of G. That is, there is a minimum spanning tree of G
  that does not include $e$.
\end{problem}

\end{document}

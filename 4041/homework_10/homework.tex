\documentclass[12pt]{article}

\usepackage[margin=1in]{geometry}
\usepackage{amsmath,amsthm,amssymb}
\usepackage{fancyhdr}
\usepackage[small,compact]{titlesec}
\usepackage{float}

\lhead{Erich Menge}
\chead{\classnameandsection}
\rhead{\homeworktitle}

\pagestyle{fancy}

\newcommand{\sethomeworknumber}[1]{
  \newcommand{\homeworktitle}{Homework #1}
}

\newcommand{\N}{\mathbb{N}}
\newcommand{\Z}{\mathbb{Z}}
\newcommand{\homeworkheader}[1]{
  \title{\vspace{2in}\homeworktitle}
  \author{Erich Menge (X.500: menge053, Student ID: 4624713) \\
  #1}
  \maketitle
  \newpage
}

\newenvironment{problem}[1]{
  \ignorespaces
  \section*{Problem #1}
}{
  \ignorespacesafterend
}

\newenvironment{solution}{
  \ignorespaces
  \subsection*{Solution}
}{
  \ignorespacesafterend
}

\newcommand{\classnameandsection}{CSCI 4011 Formal Languages And Automata Theory Section 3}


\sethomeworknumber{10}

\begin{document}
\homeworkheader{\classnameandsection}

\begin{problem}{25.1-2}
  Why do we require that $w_{ii} = 0$ for all $1 \le i \le n$?
  \begin{solution}
    Because $w_{ii}$ would be a self loop. Defining it to equal zero means that ``staying put'' doesn't incur a cost. If
    $w_{ii}$ was greater than zero you could follow the self loop an a particular node in a path an arbitrary amount of
    times and increase the cumulative weight of the path.  In other words, if it doesn't equal zero the sum of the
    weights along any given path could be arbitrarily increased. Defining $w_{ii} = 0$ allows us to safely ignore self
    loops.
  \end{solution}
\end{problem}

\begin{problem}{25.1-3}
  What does the matrix
  \[
   L^{(0)} =
   \begin{pmatrix}
    0 & \infty & \infty & \cdots & \infty\\
    \infty & 0 & \infty & \cdots & \infty\\
    \infty & \infty & 0 & \cdots & \infty\\
    \vdots & \vdots & \vdots & \ddots & \vdots \\
    \infty & \infty & \infty & \cdots & 0 \\
   \end{pmatrix}
  \]
  used in the shortest-paths algorithms correspond to in regular matrix multiplication?
  \begin{solution}
    This is the identity matrix. We have zeros down the main diagonal so our operations don't create any new self loops,
    and when this matrix is used on another matrix, the minimum between the existing matrix and this identity matrix
    will be the existing matrix, because the minimum of some weight and infinity, is the existing weight.
  \end{solution}
\end{problem}

\begin{problem}{25.1-4}
  Show that matrix multiplication defined by EXTEND-SHORTEST-PATHS is associative.
  \begin{solution}
    The main operation in extend-shortest-paths is the minimum operation. $l'_{ij}$ is initialized to $\infty$ and then $l'_{ij}$ is compared to each $l_{ik} + w_{kj}$ for all $k$ in $1\ldots n$.

    For any parenthesization of matrices the result of the minimum operation is unchanged. For example if we have some
    edge weights in the first matrix, $(a, b)$ where $a < b$, and some corresponding edge weights in a second matrix
    $(c, d)$ with $c < d$, and finally some corresponding edge weights in a third matrix $(e, f)$ with $g < h$ then
    $min(a,b) + min(c,d) + min(e, f) = a + c + e$. Also $min(a + c + e, b + d + f) = a + c + e$.  Since addition is
    associative, so is this matrix multiplication as defined by EXTEND-SHORTEST-PATHS.
  \end{solution}
\end{problem}

\begin{problem}{25.1-6}
Suppose we also wish to compute the vertices on shortest paths in the algorithms of this section. Show how to compute
the predecessor matrix $\Pi$ from the completed matrix $L$ of shortest-path weights in $O(n^3)$ time.
\end{problem}

\begin{problem}{25.1-7}
  We can also compute the vertices on shortest paths as we compute the shortest path weights.  Define $\pi^{(m)}_{ij}$
  as the predecessor of vertex $j$ on any minimum-weight path from $i$ to $j$ that contains at most $m$ edges.  Modify
  the Extend-Shortest-Paths and Slow-All-Pairs-Shortest-Paths procedures to compute the matrices
  $\Pi^{(1)}$,$\Pi^{(2)}$,$\ldots$,$\Pi^{(n-1)}$ as the matrices $L^{(1)},L^{(2)},\ldots,L^{(n-1)}$ are computed.
\end{problem}

\end{document}

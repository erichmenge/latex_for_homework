\documentclass[12pt]{article}

\usepackage[margin=1in]{geometry}
\usepackage{amsmath,amsthm,amssymb}

\newcommand{\N}{\mathbb{N}}
\newcommand{\Z}{\mathbb{Z}}
\newcommand{\homeworkheader}[2]{
  \title{Homework #1}
  \author{Erich Menge (X.500: menge053, Student ID: 4624713) \\
  #2}
  \maketitle
}

\newenvironment{problem}[1]{
  \ignorespaces
  \section*{Problem #1}
}{
  \ignorespacesafterend
}

\newcommand{\classnameandsection}{CSCI 4707 Practice of Database Systems}


\sethomeworknumber{10}

\begin{document}
\homeworkheader{\classnameandsection}

\begin{problem}{25.1-2}
  Why do we require that $w_{ii} = 0$ for all $1 \le i \le n$?
\end{problem}

\begin{problem}{25.1-3}
What does the matrix
\[
 L^{(0)} =
 \begin{pmatrix}
  0 & \infty & \infty & \cdots & \infty\\
  \infty & 0 & \infty & \cdots & \infty\\
  \infty & \infty & 0 & \cdots & \infty\\
  \vdots & \vdots & \vdots & \ddots & \vdots \\
  \infty & \infty & \infty & \cdots & 0 \\
 \end{pmatrix}
\]
used in the shortest-paths algorithms correspond to in regular matrix multiplication?
\end{problem}

\begin{problem}{25.1-4}
  Show that matrix multiplication defined by EXTEND-SHORTEST-PATHS is associative.
\end{problem}

\begin{problem}{25.1-6}
Suppose we also wish to compute the vertices on shortest paths in the algorithms of this section. Show how to compute
the predecessor matrix $\Pi$ from the completed matrix $L$ of shortest-path weights in $O(n^3)$ time.
\end{problem}

\begin{problem}{25.1-7}
  We can also compute the vertices on shortest paths as we compute the shortest path weights.  Define $\pi^{(m)}_{ij}$
  as the predecessor of vertex $j$ on any minimum-weight path from $i$ to $j$ that contains at most $m$ edges.  Modify
  the Extend-Shortest-Paths and Slow-All-Pairs-Shortest-Paths procedures to compute the matrices
  $\Pi^{(1)}$,$\Pi^{(2)}$,$\ldots$,$\Pi^{(n-1)}$ as the matrices $L^{(1)},L^{(2)},\ldots,L^{(n-1)}$ are computed.
\end{problem}

\end{document}

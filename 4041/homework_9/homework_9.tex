\documentclass[12pt]{article}

\usepackage[margin=1in]{geometry}
\usepackage{amsmath,amsthm,amssymb}
\usepackage{fancyhdr}
\usepackage[small,compact]{titlesec}
\usepackage{float}

\lhead{Erich Menge}
\chead{\classnameandsection}
\rhead{\homeworktitle}

\pagestyle{fancy}

\newcommand{\sethomeworknumber}[1]{
  \newcommand{\homeworktitle}{Homework #1}
}

\newcommand{\N}{\mathbb{N}}
\newcommand{\Z}{\mathbb{Z}}
\newcommand{\homeworkheader}[1]{
  \title{\vspace{2in}\homeworktitle}
  \author{Erich Menge (X.500: menge053, Student ID: 4624713) \\
  #1}
  \maketitle
  \newpage
}

\newenvironment{problem}[1]{
  \ignorespaces
  \section*{Problem #1}
}{
  \ignorespacesafterend
}

\newenvironment{solution}{
  \ignorespaces
  \subsection*{Solution}
}{
  \ignorespacesafterend
}

\newcommand{\classnameandsection}{CSCI 4011 Formal Languages And Automata Theory Section 3}


\sethomeworknumber{9}

\begin{document}
\homeworkheader{\classnameandsection}

\begin{problem}{23.1-3}
  Show that if an edge $(u,v)$ is contained in some minimum spanning tree, then it is a light edge crossing some cut of
  the graph.
  \begin{solution}
    By definition a light edge crossing the cut is a minimum weight edge. When we have some graph $G$ we can expand a
    minimum spaning tree such that each edge of the spanning tree is the minimum weight edge, or equal to some minimum
    weight edge of G.

    Since each edge of the minimum spanning tree maintains this invariant where $A$ is a subset of edges of the minimum
    spanning tree we know that the edge $(u,v)$ exists in that set $A$ which maintains the invariant. That means that
    there must exist some cut in which edge $(u,v)$ is a light edge. If it wasn't, the invariant wouldn't hold for that
    edge (there would be some other edge that would be a lighter weight, violating the definition of a minimum spanning
    tree).  Thus $(u,v)$ must be a light edge in some cut of the graph.
  \end{solution}
\end{problem}

\begin{problem}{23.1-7}
  Argue that if all edge weights of a graph are positive, then any subset of edges that connects all vertices and has
  minimum total weight must be a tree. Give an example to show that the same conclusion does not follow if we allow some
  weights to be nonpositive.
\end{problem}

\begin{problem}{23.1-8}
  Let T be a minimum spanning tree of a graph G, and let L be the sorted list of the edge weights of T. Show that for
  any other minimum spanning tree $T'$ of G, the list L is also the sorted list of edge weights of $T'$.
\end{problem}

\begin{problem}{23.1-9}
  Let T be a minimum spanning tree of a graph $G = (V,E)$, and let $V'$ be a subset of $V$. Let $T'$ be the subgraph of
  $T$ induced by $V'$, and let $G'$ be the subgraph of $G$ induced by $V'$. Show that if $T'$ is connected, then $T'$ is
  a minimum spanning tree of $G'$.
\end{problem}

\begin{problem}{23.2-1}
  Kruskal's algorithm can return different spanning trees for the same input graph G, depending on how it breaks ties
  when the edges are sorted into order. Show that for each minimum spanning tree $T$ of $G$, there is a way to sort the
  edges of $G$ in Kruskal's algorithm so that the algorithm returns $T$.
\end{problem}

\begin{problem}{23.2-2}
  Suppose that we represent the graph $G = (V,E)$ as an adjacency matrix. Give a simple implementation of Prim's
  algorithm for this case that runs in $O(V^2)$ time.
  \begin{solution}
    We can use a modified version of the algorithm in the book. Since in an adjacency matrix we have $|V|$ rows and $|V|$
    columns it is a $|V| \times |V|$ matrix. The main loop iterates over all the verticies, so that is $O(V)$. Then we need
    to check the corresponding row in the adjacency matrix, which has $|V|$ columns. This matrix check is done in each of
    the main loops, so it is done $V$ times. So we have $O(V^2)$. Note that only the row needs to be checked because it
    is not directed, and thus the adjacency matrix is symmetric so there is no need to check the corresponding column as
    well.
    \begin{lstlisting}[mathescape]
      MST-Prim(G, w, r)
        for each $u \in$ G.V
          u.key = $\infty$
          u.$\pi$ = NIL
        r.key = 0
        Q = G.V
        while Q $\ne$ $\emptyset$
          u = Extract-Min(Q)
          for each v $\in$ G.AdjMatrix[G.rowof(u)][0..|V| - 1]
            if v $\in$ Q and w(u,v) < v.key
              v.$\pi$ = u
              v.key = w(u,v)
    \end{lstlisting}
  \end{solution}
\end{problem}

\end{document}

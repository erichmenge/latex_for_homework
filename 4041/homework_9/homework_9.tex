\documentclass[12pt]{article}

\usepackage[margin=1in]{geometry}
\usepackage{amsmath,amsthm,amssymb}
\usepackage{fancyhdr}
\usepackage[small,compact]{titlesec}
\usepackage{float}

\lhead{Erich Menge}
\chead{\classnameandsection}
\rhead{\homeworktitle}

\pagestyle{fancy}

\newcommand{\sethomeworknumber}[1]{
  \newcommand{\homeworktitle}{Homework #1}
}

\newcommand{\N}{\mathbb{N}}
\newcommand{\Z}{\mathbb{Z}}
\newcommand{\homeworkheader}[1]{
  \title{\vspace{2in}\homeworktitle}
  \author{Erich Menge (X.500: menge053, Student ID: 4624713) \\
  #1}
  \maketitle
  \newpage
}

\newenvironment{problem}[1]{
  \ignorespaces
  \section*{Problem #1}
}{
  \ignorespacesafterend
}

\newenvironment{solution}{
  \ignorespaces
  \subsection*{Solution}
}{
  \ignorespacesafterend
}

\newcommand{\classnameandsection}{CSCI 4011 Formal Languages And Automata Theory Section 3}


\sethomeworknumber{9}

\begin{document}
\homeworkheader{\classnameandsection}

\begin{problem}{23.1-3}
  Show that if an edge $(u,v)$ is contained in some minimum spanning tree, then it is a light edge crossing some cut of
  the graph.
\end{problem}

\begin{problem}{23.1-7}
  Argue that if all edge weights of a graph are positive, then any subset of edges that connects all vertices and has
  minimum total weight must be a tree. Give an example to show that the same conclusion does not follow if we allow some
  weights to be nonpositive.
\end{problem}

\begin{problem}{23.1-8}
  Let T be a minimum spanning tree of a graph G, and let L be the sorted list of the edge weights of T. Show that for
  any other minimum spanning tree $T'$ of G, the list L is also the sorted list of edge weights of $T'$.
\end{problem}

\begin{problem}{23.1-9}
  Let T be a minimum spanning tree of a graph $G = (V,E)$, and let $V'$ be a subset of $V$. Let $T'$ be the subgraph of
  $T$ induced by $V'$, and let $G'$ be the subgraph of $G$ induced by $V'$. Show that if $T'$ is connected, then $T'$ is
  a minimum spanning tree of $G'$.
\end{problem}

\begin{problem}{23.2-1}
  Kruskal's algorithm can return different spanning trees for the same input graph G, depending on how it breaks ties
  when the edges are sorted into order. Show that for each minimum spanning tree $T$ of $G$, there is a way to sort the
  edges of $G$ in Kruskal's algorithm so that the algorithm returns $T$.
\end{problem}

\begin{problem}{23.2-2}
  Suppose that we represent the graph $G = (V,E)$ as an adjacency matrix. Give a simple implementation of Prim's
  algorithm for this case that runs in $O(V^2)$ time.
\end{problem}

\end{document}

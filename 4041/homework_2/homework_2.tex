\documentclass[12pt]{article}

\usepackage[margin=1in]{geometry}
\usepackage{amsmath,amsthm,amssymb}
\usepackage{fancyhdr}
\usepackage[small,compact]{titlesec}
\usepackage{float}

\lhead{Erich Menge}
\chead{\classnameandsection}
\rhead{\homeworktitle}

\pagestyle{fancy}

\newcommand{\sethomeworknumber}[1]{
  \newcommand{\homeworktitle}{Homework #1}
}

\newcommand{\N}{\mathbb{N}}
\newcommand{\Z}{\mathbb{Z}}
\newcommand{\homeworkheader}[1]{
  \title{\vspace{2in}\homeworktitle}
  \author{Erich Menge (X.500: menge053, Student ID: 4624713) \\
  #1}
  \maketitle
  \newpage
}

\newenvironment{problem}[1]{
  \ignorespaces
  \section*{Problem #1}
}{
  \ignorespacesafterend
}

\newenvironment{solution}{
  \ignorespaces
  \subsection*{Solution}
}{
  \ignorespacesafterend
}

\newcommand{\classnameandsection}{CSCI 4011 Formal Languages And Automata Theory Section 3}


\sethomeworknumber{2}

\begin{document}

\homeworkheader{\classnameandsection}

\begin{problem}{6.5-7}
  Show how to implement a first-in, first-out queue with a priority queue. Show how to implement a stack with
  a priority queue.
\end{problem}

\begin{problem}{6.5-8}
  The operation HEAP-DELETE(A,i) deletes the item in node i from heap A. Give an implementation of HEAP-DELETE that runs
  in $O(lgn)$ time for an n-element max-heap.
\end{problem}

\begin{problem}{6.5-9}
  Give an $O(nlgk)$-time algorithm to merge k sorted lists into one sorted list, where n is the total number of elements in
  all the input lists. (Hint: Use a min- heap for k-way merging.)
\end{problem}

\begin{problem}{2.3-2}
  Rewrite the MERGE procedure so that it does not use sentinels, instead stopping once either array L or R has had all its
  elements copied back to A and then copying the remainder of the other array back into A.
\end{problem}

\begin{problem}{2.3-3}
Use mathematical induction to show that when $n$ is an exact power of 2, the solution of the recurrence
  $T(n) = \begin{cases}
    2 & \text{if } n = 2 \\
    2T(n/2) + n & \text{if } n = 2^k, \text{for } k >1
  \end{cases}
  $
\end{problem}

\begin{problem}{2.3-4}
  We can express insertion sort as a recursive procedure as follows. In order to sort $A[1\ldots n]$ we recursively sort
  $A[1\ldots n] - 1 $ and then insert $A[n]$ into the sorted array $A[1\ldots n] - 1 $.
  Write a recurrence for the running time of this recursive version of insertion sort.
\end{problem}

\end{document}

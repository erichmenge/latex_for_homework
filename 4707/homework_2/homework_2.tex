\documentclass[12pt]{article}

\usepackage[margin=1in]{geometry}
\usepackage{amsmath,amsthm,amssymb}
\usepackage{fancyhdr}
\usepackage[small,compact]{titlesec}
\usepackage{float}

\lhead{Erich Menge}
\chead{\classnameandsection}
\rhead{\homeworktitle}

\pagestyle{fancy}

\newcommand{\sethomeworknumber}[1]{
  \newcommand{\homeworktitle}{Homework #1}
}

\newcommand{\N}{\mathbb{N}}
\newcommand{\Z}{\mathbb{Z}}
\newcommand{\homeworkheader}[1]{
  \title{\vspace{2in}\homeworktitle}
  \author{Erich Menge (X.500: menge053, Student ID: 4624713) \\
  #1}
  \maketitle
  \newpage
}

\newenvironment{problem}[1]{
  \ignorespaces
  \section*{Problem #1}
}{
  \ignorespacesafterend
}

\newenvironment{solution}{
  \ignorespaces
  \subsection*{Solution}
}{
  \ignorespacesafterend
}

\newcommand{\classnameandsection}{CSCI 4011 Formal Languages And Automata Theory Section 3}

\lstset{language=SQL,breaklines=true}

\sethomeworknumber{2}

\begin{document}
\homeworkheader{\classnameandsection}

\begin{problem}{3.12}
  Consider the scenario from Exercise 2.2, where you designed an ER diagram for a university database. Write SQL
  statements to create the corresponding relations and capture as many of the constraints as possible. If you cannot:
  capture some constraints, explain why.
  \begin{solution}
    \textbf{SQL Statements for 2.2-1}
    \lstinputlisting{problem_3_12_1.sql}
    \textbf{SQL Statements for 2.2-2}
    \lstinputlisting{problem_3_12_2.sql}
    \textbf{SQL Statements for 2.2-3} \\
    We can't check to ensure every professor teaches some class unless we were to check this when the professors are
    inserted.
    \br
    \textbf{SQL Statements for 2.2-4}
    \lstinputlisting{problem_3_12_4.sql}
    \textbf{SQL Statements for 2.2-5}
    \lstinputlisting{problem_3_12_5.sql}
    \textbf{SQL Statements for 2.2-6}
    \lstinputlisting{problem_3_12_6.sql}
  \end{solution}
\end{problem}

\begin{problem}{3.14}
  Consider the scenario from Exercise 2.4, where you designed an ER diagram for a company database write SQL statements
  to create the corresponding relations and capture as many of the constraints as possible. If you cannot capture some
  constraints, explain why.
  \begin{solution}
    \lstinputlisting{problem_3_14.sql}
  \end{solution}
\end{problem}

\begin{problem}{4.2}
  \begin{solution}
    1. Minimum = N2, Maximum = N1 + N2, Requires union compatibility\\
    2. Minimum = 0, Maximum = N1, Requires union compatibility\\
    3. Minimum = 0, Maximum = N1, Requires union compatibility \\
    4. Minimum = Maximum = $N1 \times N2$ \\
    5. Minimum = 0, Maximum = N1, Assumption: There is some attribute $a$ \\
    6. Minimum = 0, Maximum = N1, Assumption: There is some attribute $a$ \\
    7. Minimum = Maximum = 0, Assumption: R2 is a subset of R1.
  \end{solution}
\end{problem}

\begin{problem}{4.3(Relational Algebra ONLY)}
\end{problem}

\begin{problem}{4.4}
  \begin{solution}
    1. Give the names of suppliers selling a red part for less than \$100.00.
    \br
    2. sid is projected, then sname is projected. This relational algebra query doesn't return anything.
    \br
    3. This computes the names of suppliers that sell a red part and green part for less than \$100.00
    \br
    4. This computes the supplier ids of suppliers that sell a red part and green part for less than \$100.00
    \br
    5. This computes the names of suppliers that sell a red part and green part for less than \$100.00
  \end{solution}
\end{problem}

\begin{problem}{5.2}
  1. \begin{lstlisting}
    SELECT DISTINCT pnames FROM catalog, parts WHERE catalog.pid = parts.pid
  \end{lstlisting}
  2. \begin{lstlisting}
    SELECT suppliers.sname FROM suppliers WHERE suppliers.sname NOT IN(
      (SELECT parts.pid FROM parts)
      MINUS
      (SELECT catalog.pid FROM catalog WHERE catalog.pid = parts.pid)
    )
  \end{lstlisting}
  3. \begin{lstlisting}
    SELECT suppliers.sname FROM suppliers WHERE suppliers.sname NOT IN(
      (SELECT parts.pid FROM parts WHERE parts.color = red)
      MINUS
      (SELECT catalog.pid FROM catalog,parts WHERE catalog.pid = parts.pid AND parts.color = red)
    )
  \end{lstlisting}
  4. \begin{lstlisting}
    SELECT pnames FROM Parts P, Suppliers S, Catalog C WHERE
      S.sname = 'Acme Widget Suppliers'
      AND S.sid = C.sid
      AND C.pid = P.pid
  \end{lstlisting}
  5. \begin{lstlisting}
    SELECT C.sid FROM Catalog C
    WHERE C.cost > (
      SELECT average(CC.cost) FROM Catalog CC
      WHERE C.pid = CC.pid
    )
  \end{lstlisting}
  6. \begin{lstlisting}
    SELECT S.sname, FROM Suppliers S, Parts P, Catalog C
    WHERE C.pid = P.pid AND
          S.sid = C.sid AND
          C.cost = (
            SELECT max(CC.cost) FROM Catalog CC WHERE CC.pid = P.pid
          )
  \end{lstlisting}
  7. \begin{lstlisting}
    SELECT DISTINCT C.sid FROM Parts P, Catalog C
    WHERE C.pid = P.pid AND
          P.color = 'red'
  \end{lstlisting}
  8. \begin{lstlisting}
    SELECT C.sid FROM Catalog C, Parts P
      WHERE P.color = "red" AND
            P.pid = C.pid
    INTERSECT
    SELECT C.sid FROM Catalog C, Parts P
      WHERE P.color = "green" AND
            P.pid = C.pid
  \end{lstlisting}
  9. \begin{lstlisting}
      SELECT C.sid FROM Catalog C, Parts P
        WHERE P.color = "red" AND
              P.pid = C.pid
      UNION
      SELECT C.sid FROM Catalog C, Parts P
        WHERE P.color = "green" AND
              P.pid = C.pid
  \end{lstlisting}
  10.
  \begin{lstlisting}
    SELECT S.sname, count(*) AS PC FROM Suppliers S, Parts P, Catalog C
      WHERE P.pid = C.pid AND
            C.sid = S.sid
      GROUP BY S.sname
      HAVING P.color = 'green'
  \end{lstlisting}
  11.
  \begin{lstlisting}
    SELECT S.sname, max(C.cost) FROM (
      SELECT S.sname, C.cost FROM Suppliers S, Catalog C, Parts P
        WHERE P.pid = C.pid AND C.sid = S.sid
        GROUP BY S.sname
        HAVING p.color = 'green'
      INTERSECT
      SELECT S.sname, C.cost FROM Suppliers S, Catalog C, Parts P
        WHERE P.pid = C.pid AND C.sid = S.sid
        GROUP BY S.sname
        HAVING p.color = 'red'
    )
    GROUP BY S.sname
  \end{lstlisting}
\end{problem}

\begin{problem}{5.4}
\end{problem}

\end{document}

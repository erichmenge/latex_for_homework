\documentclass[12pt]{article}

\usepackage[margin=1in]{geometry}
\usepackage{amsmath,amsthm,amssymb}
\usepackage{fancyhdr}
\usepackage[small,compact]{titlesec}
\usepackage{float}

\lhead{Erich Menge}
\chead{\classnameandsection}
\rhead{\homeworktitle}

\pagestyle{fancy}

\newcommand{\sethomeworknumber}[1]{
  \newcommand{\homeworktitle}{Homework #1}
}

\newcommand{\N}{\mathbb{N}}
\newcommand{\Z}{\mathbb{Z}}
\newcommand{\homeworkheader}[1]{
  \title{\vspace{2in}\homeworktitle}
  \author{Erich Menge (X.500: menge053, Student ID: 4624713) \\
  #1}
  \maketitle
  \newpage
}

\newenvironment{problem}[1]{
  \ignorespaces
  \section*{Problem #1}
}{
  \ignorespacesafterend
}

\newenvironment{solution}{
  \ignorespaces
  \subsection*{Solution}
}{
  \ignorespacesafterend
}

\newcommand{\classnameandsection}{CSCI 4011 Formal Languages And Automata Theory Section 3}


\sethomeworknumber{2}

\begin{document}
\homeworkheader{\classnameandsection}

\begin{problem}{3.12}
  Consider the scenario from Exercise 2.2, where you designed an ER diagram for a university database. Write SQL
  statements to create the corresponding relations and capture as many of the constraints as possible. If you cannot:
  capture some constraints, explain why.
  \begin{solution}
    \textbf{SQL Statements for 2.2-1}
    \lstinputlisting{problem_3_12_1.sql}
    \textbf{SQL Statements for 2.2-2}
    \lstinputlisting{problem_3_12_2.sql}
    \textbf{SQL Statements for 2.2-3} \\
    We can't check to ensure every professor teaches some class unless we were to check this when the professors are
    inserted.
    \br
    \textbf{SQL Statements for 2.2-4}
    \lstinputlisting{problem_3_12_4.sql}
    \textbf{SQL Statements for 2.2-5}
    \lstinputlisting{problem_3_12_5.sql}
    \textbf{SQL Statements for 2.2-6}
    \lstinputlisting{problem_3_12_6.sql}
  \end{solution}
\end{problem}

\begin{problem}{3.14}
  Consider the scenario from Exercise 2.4, where you designed an ER diagram for a company database write SQL statements
  to create the corresponding relations and capture as many of the constraints as possible. If you cannot capture some
  constraints, explain why.
  \begin{solution}
    \lstinputlisting{problem_3_14.sql}
  \end{solution}
\end{problem}

\begin{problem}{4.2}
  \begin{solution}
    1. Minimum = N2, Maximum = N1 + N2, Requires union compatibility\\
    2. Minimum = 0, Maximum = N1, Requires union compatibility\\
    3. Minimum = 0, Maximum = N1, Requires union compatibility \\
    4. Minimum = Maximum = $N1 \times N2$ \\
    5. Minimum = 0, Maximum = N1, Assumption: There is some attribute $a$ \\
    6. Minimum = 0, Maximum = N1, Assumption: There is some attribute $a$ \\
    7. Minimum = Maximum = 0, Assumption: R2 is a subset of R1.
  \end{solution}
\end{problem}

\begin{problem}{4.3(Relational Algebra ONLY)}
\end{problem}

\begin{problem}{4.4}
\end{problem}

\begin{problem}{5.2}
\end{problem}

\begin{problem}{5.4}
\end{problem}

\end{document}

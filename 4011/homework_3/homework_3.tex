\documentclass[12pt]{article}

\usepackage[margin=1in]{geometry}
\usepackage{amsmath,amsthm,amssymb}
\usepackage{fancyhdr}
\usepackage[small,compact]{titlesec}
\usepackage{float}

\lhead{Erich Menge}
\chead{\classnameandsection}
\rhead{\homeworktitle}

\pagestyle{fancy}

\newcommand{\sethomeworknumber}[1]{
  \newcommand{\homeworktitle}{Homework #1}
}

\newcommand{\N}{\mathbb{N}}
\newcommand{\Z}{\mathbb{Z}}
\newcommand{\homeworkheader}[1]{
  \title{\vspace{2in}\homeworktitle}
  \author{Erich Menge (X.500: menge053, Student ID: 4624713) \\
  #1}
  \maketitle
  \newpage
}

\newenvironment{problem}[1]{
  \ignorespaces
  \section*{Problem #1}
}{
  \ignorespacesafterend
}

\newenvironment{solution}{
  \ignorespaces
  \subsection*{Solution}
}{
  \ignorespacesafterend
}

\newcommand{\classnameandsection}{CSCI 4011 Formal Languages And Automata Theory Section 3}


\sethomeworknumber{3}

\begin{document}
\homeworkheader{\classnameandsection}

\begin{problem}{1}
  a. $\{w \in \{a,b\}^*\ |\ \text{w has twice as many bs as as} \}$
  \br
  b. Balanced parentheses and square brackets. Thus, ()(), [][] and (([])[()]) would be legal expressions of this
     language but ([)] and [(()] would not.
  \br
  c. The language in Exercise 2.6, part d, in the book.
  \br
  d. $\{ a^ib^jc^k | i = j or j = k \}$
  \begin{solution}
    a. At first I was tempted to do something like $S \rightarrow bab | bba | abb | A | \epsilon$ but that wouldn't
    allow for long chains such as $aabaabbbbbbb$. \\ So instead it should be $S \rightarrow bSaSb | bSbSa |
    aSbSb|\epsilon$. Which is much like above except it separates each character with $S$ to allow enough flexibility
    for the $\Sigma^*$ requirement.
    \br
    b. $S \rightarrow (S)\ |\ [S]\ |\ \epsilon$
    \br
    c.
    \br
    d.
    \begin{align*}
      S &\rightarrow BC'|AC \\
      B &\rightarrow aBb | \epsilon \\
      C &\rightarrow bCc | \epsilon \\
      C' &\rightarrow cC' | \epsilon \\
      A &\rightarrow aA | \epsilon
    \end{align*}
    So this says from our start position we can have balanced $a$s and $b$s followed by any number of $c$s, or we can
    have any number of $a$s followed by balanced $b$s and $c$s.
  \end{solution}
\end{problem}
\end{document}

\documentclass[12pt]{article}

\usepackage[margin=1in]{geometry}
\usepackage{amsmath,amsthm,amssymb}

\newcommand{\N}{\mathbb{N}}
\newcommand{\Z}{\mathbb{Z}}
\newcommand{\homeworkheader}[2]{
  \title{Homework #1}
  \author{Erich Menge (X.500: menge053, Student ID: 4624713) \\
  #2}
  \maketitle
}

\newenvironment{problem}[1]{
  \ignorespaces
  \section*{Problem #1}
}{
  \ignorespacesafterend
}

\newcommand{\classnameandsection}{CSCI 4707 Practice of Database Systems}


\sethomeworknumber{4}

\begin{document}
\homeworkheader{\classnameandsection}

\begin{problem}{1}
  \begin{solution}
    $M =$ ``
    \begin{enumerate}
      \item Scan the tape starting at the beginning from left to right, if no $1$s or $0$s are found, accept.\\
      \item Move the head back to the beginning. Scan from left to right looking for a $1$, if none is found, reject, else cross it off and move back to the beginning of the tape.\\
      \item Scan from left to right looking for a zero, if none is found, reject, otherwise cross it off and continue.\\
      \item Continue scanning from left to right looking for a zero, if none is found, reject, otherwise cross it off and
      move the head back to the beginning.\\
      \item Repeat the above steps.
    \end{enumerate}
    ''
  \end{solution}
\end{problem}

\begin{problem}{2}
  \begin{solution}
    \begin{align*}
      M &= (Q, \Sigma, \Gamma, \delta, q_0, q_9,q_{\text{reject}}) \\
      Q &= \{ q_0,q_1,q_2,q_3,q_4,q_5,q_6,q_7,q_8,q_9 \} \\
      \Sigma &= \{ 0, 1 \}\\
      \Gamma &= \{ x, \$ \} \cup \Sigma \\
      \delta &\text{ is shown in the figure below with the state diagram.}
    \end{align*}
    \begin{figure}[H]
      \centering
      \caption{State Diagram}
      \includegraphics[scale=.6]{problem_2.png}
    \end{figure}
    Initially if the tape is blank accept, as zero $0$s is twice as many as zero $1$s. Otherwise, append a \$ onto the
    tape so the machine knows where the beginning of the tape is. It can then easily mark off pairs of 0s for each 1 it
    encounters. After each cycle, it rewinds the tape to the \$ so that it starts at the beginning.
  \end{solution}
\end{problem}

\begin{problem}{3}
  \begin{solution}

  \end{solution}
\end{problem}

\begin{problem}{4}
  \begin{solution}

  \end{solution}
\end{problem}

\begin{problem}{5}
  \begin{solution}

  \end{solution}
\end{problem}

\begin{problem}{6}
  \begin{solution}

  \end{solution}
\end{problem}

\begin{problem}{7}
  \begin{solution}

  \end{solution}
\end{problem}

\begin{problem}{8}
  \begin{solution}

  \end{solution}
\end{problem}

\end{document}
